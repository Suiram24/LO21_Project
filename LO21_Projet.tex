\documentclass{article}
\usepackage[frenchb]{babel}
\usepackage[T1]{fontenc}
\usepackage{times}
\usepackage[utf8]{inputenc}
\usepackage{amsmath}
\usepackage{textcomp}
\usepackage{mathtools}
\usepackage{arevmath}
\usepackage{amsfonts}
\usepackage{slantsc}
\usepackage{xcolor}
\usepackage[a4paper, margin=1in]{geometry}

\usepackage[linesnumbered, french]{algorithm2e}
\SetKwInput{Data}{Donn\'ees}
\SetKwInput{Result}{R\'esultat}
\SetKwInput{Vars}{Variables}
\SetKwInput{Assertion}{Assertion}
\SetKwIF{If}{ElseIf}{Else}{Si}{alors}{Sinon si}{Sinon}{FinSi}
\SetKwFor{While}{Tant que}{faire}{Fin TantQue}
\SetKwBlock{Begin}{D\'ebut}{Fin}
\SetKwRepeat{Repeat}{Faire}{Tant que}
\DontPrintSemicolon
\SetAlgoLined

\RestyleAlgo{boxruled}

\newcommand{\Assign}[2]{#1 $\; \longleftarrow \;$ #2}
\newcommand{\Arg}[2]{\hspace{0.2em}#1\hspace{0.05em}: #2}
\newcommand{\Child}[2]{#2(#1)}
\newcommand{\Null}[0]{\textsc{null}\hspace{4pt}}
\newcommand{\Et}[2]{#1\hspace{2pt}\textnormal{\textbf{et}}\hspace{2.5pt}#2}
\newcommand{\Not}[1]{\textnormal{\textbf{non}}(#1)}

\begin{document}

\title{LO21}
\author{Marius Bozane\\
Louis Esteban
}
\date {Automne 2020}
\maketitle

\begin{abstract}
	Pour notre projet de LO21 nous devions réaliser un système expert composé de bases de 	connaissances, de règles et d'un moteur d'inférence.
\end{abstract}

\newpage

\section{Règles}

\subsection{Création de règle}
On commmence par créer une nouvelle règle de type pointeur\\
\begin{algorithm}[H]

\Vars{\Arg{$R$}{La nouvelle Règle}}

\Result{\Arg{$R$}{La Règle à retourner}}

\Begin({RègleVide()}){
	\Assign{$R$}{\textit{règle\_vide}}
}
\caption{RègleVide\label{RV}}

\end{algorithm}

\subsection{Ajout de Prémisses}

\begin{algorithm}[H]

\Vars{\\
- \Arg{$P$}{Prémisse à rajouter}\\
- \Arg{$R$}{Règle dont on veut rajouter une prémisse}\\
- \Arg{$R'$}{Stockage du dernier objet d'une règle}\\
- \Arg{$T$}{Stockage de l'avant-dernier objet de la règle}\\
- \Arg{$NP$}{L'adresse de la nouvelle prémisse}}

\Data{\Arg{$R$}{Règle}}

\Result{\Arg{$R$}{Règle}}

\Begin({CréerConclusion ($C$,$R$)}){

\While{Suivant($R$) $\neq$ indéfini}{
	
	\Assign{$T$}{Suivant($R$)}\\
	\Assign{$R$}{Suivant($R$)}
	
}
\Assign{Conclusion($R'$)}{$R$}\\
\Assign{Fait($NP$)}{$R$}

\If{Conclusion($R'$)=1}{
	\Assign{Suivant($T$)}{$NP$}\\
	\Assign{Suivant($NP$)}{$R'$}
} \Else {
	\Assign{Suivant($R$)}{$NP$}
}
}

\caption{AjoutPrémisse\label{AP}}

\end{algorithm}

\subsection{Créer une conclusion}

\begin{algorithm}[H]
\Vars{\\

- \Arg{$C$}{Conclusion à rajouter}\\
- \Arg{$R$}{Règle dont on veut rajouter une conclusion}\\
- \Arg{$R'$}{Règle de transit}}

\Data{\Arg{$R$}{Règle}}
\Result{\Arg{$R$}{La règle auquelle on veut rajouter une conclusion}}

\Begin({CréerConclusion ($C$,$R$)}){

\While{Suivant($R$) $\neq$ indéfini}{

	\Assign{$R$}{Suivant($R$)}
	
}
\If{Conclusion($R$) = 0}{

	\Assign{$R'$}{\textit{RègleVide()}}\\
	\Assign{Fait($R'$)}{$C$}\\
	\Assign {Conclusion($R'$)}{1}\\
	\Assign{Suivant($R$)}{$R'$}\\
	\Assign{Résultat}{Vrai}
	
}
\Else {
	\Assign{Résultat}{Faux}
}
}
\caption{CréerConclusion\label{CCL}}

\end{algorithm}

\subsection{Test 1: Une prémisse appartient à une règle}

\begin{algorithm}[H]

\Vars{\\
- \Arg{$R$}{Règle}\\
- \Arg{$P$}{Prémisse que l'on veut tester}}
\Data{\Arg{$R$}{Règle}}
\Result{\Arg{$R$}{Règle}}

\Begin({TestPrémisse ($P$,$R$)}){

\While{Suivant($R$) $\neq$ indéfini}{

	\If{ComparerCaractère(Fact($R$),$P$)=0}{
	
		\Assign{Résultat}{Vrai}
		
	}
	
	\Assign{$R$}{Suivant($R$)}
	
}

\If{ \Et (Conlusion($R$)=0) (ComparerCaractère(Fact($R$),$P$)=0)}{

	\Assign{Résultat}{Vrai}
	
} \Else {
	\Assign{Résultat}{Faux}
}

}

\caption{TestPrémisse\label{TP}}

\end{algorithm}

\subsection{Supprimer une prémisse d'une règle}

\begin{algorithm}[H]

\Vars{\\
- \Arg{$P$}{Prémisse à rajouter}\\
- \Arg{$R$}{Règle dont on veut rajouter une prémisse}\\
- \Arg{$R'$}{Stockage du dernier objet d'une règle}\\
- \Arg{$T$}{Stockage de l'avant-dernier objet de la règle}\\

\Data{\Arg{$R$}{Règle}}

\Result{\Arg{$R$}{Règle}}

\Begin({SupprimerPrémisse ($P$,$R$)}){

\If {\textit{PrémisseVide($R$}=1}{

	\Assign{Résultat}{Faux}
	
}\Else{

	\While{$R$ $\neq$ indéfini}{
	
	\Assign{$T$}{$R$}\\
	
	\If{\Et(ComparerCaractère(Fait($R$),$P$)=1) (Conclusion($R$)=1)}{
	
		\Assign{Suivant($T$)}{Suivant($R$)}\\
		Libérer($R$)\\
		\Assign{Résultat}{Vrai}
	}
	\Assign{$R$}{Suivant($R$)}
	}
	\Assign{Résultat}{Faux}
}

}

}

\caption{SupprimerPrémisse\label{SP}}

\end{algorithm}

\subsection{Test 2: Prémisse vide d'une règle}

\begin{algorithm}[H]

\Vars{\Arg{$R$}{Règle que l'on veut tester}}
\Data{\Arg{$R$}{la Règle en question}}
\Result{\Arg{Résultat}{Précise si la règle contient des prémisses ou non}}

\Begin({PrémisseVide ($R$)}){

	\If{Suivant($R$) $\neq$ indéfini}{
	\Assign{Résultat}{Faux}
	}
	
	\ElseIf {Conclusion(Suivant($R$))=1}{
	\Assign{Résultat}{Vrai}	
	}
	
	\Else {
	\Assign{Résultat}{Faux}
	}
}
\caption{PrémisseVide\label{PV}}

\end{algorithm}

\subsection{Accéder à la première prémisse d'une règle}

\begin{algorithm}[H]

\Vars{\Arg{$R$}{Règle dont on veut voir la prémisse}}
\Data{\Arg{$R$}{la Règle en question}}
\Result{\Arg{$P$}{Renvoit la première prémisse}}

\Begin({PremièrePrémisse ($R$)}){

	\While{Suivant($R$) $\neq$ indéfini}{

	\Assign{$R$}{Suivant($R$)}
	}
	\If{Conclusion($R$)=1}{
		Retourner (Fait($R$))\\
	}
		Retourner indéfini\\
}


\caption{PremièrePrémisse\label{PP}}

\end{algorithm}

\subsection{Accéder à la conclusion d'une règle}

\begin{algorithm}[H]

\Vars{\Arg{$R$}{Règle dont on veut la conclusion}}

\Data{\Arg{$R$}{la Règle en question}}

\Result{\Arg{$P$}{La Conclusion de la règle si elle existe}}

\Begin({VoirConclusion ($C$,$R$)}){

\While{Suivant($R$) $\neq$ indéfini}{
	\Assign{$R$}{Suivant($R$)}	
}
\If{Conclusion(Rule)=1}{
	\Assign{$P$}{Fait($R$)}
}
\Else{
	\Assign{$P$}{indéfini}  
}
}

\caption{VoirConclusion\label{VC}}

\end{algorithm}


\section{Base de Connaissances}

\section{Moteur d'Inférence}
\end{document}